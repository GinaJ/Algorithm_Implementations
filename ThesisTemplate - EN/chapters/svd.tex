\chapter{Singular Value Decomposition}
\section{Example of a SVD computation}

We have the matrix C as:
\begin{center}
	$ 
	C=\begin{bmatrix}
	5 &   5 \\
-1  &  7 \\
	\end{bmatrix}$
\end{center}
and we want to write it as $ C=U\Sigma V^{T} $.\\
We have to take into account these 2 equations:
\begin{itemize}
	\item $ C^T C = V\Sigma^T\Sigma V^T$
	\item $CV=U\Sigma$
\end{itemize}
Let's compute $ C^T \cdot C $\\
\begin{center}
	$ C^T \cdot C =\begin{bmatrix}
5 &   -1 \\
5  &  7 \\
\end{bmatrix}\begin{bmatrix}
5 &   5 \\
-1  &  7 \\
\end{bmatrix}=\begin{bmatrix}
26 &   18 \\
18  &  74 \\
\end{bmatrix}$
\end{center}
Now we can find the eigenvalues of the matrix $ C^T \cdot C $.
We know that an eigenvalue of $ C^T \cdot C $, is a solutions of the polynomial equation $det( C^T \cdot C-\lambda I)=0 $, where $ I $ is the identity matrix.