\chapter{Introduction}
The topic discussed in this work of thesis is based on an ASIC (Application-specific integrated circuit) architecture to implement some algorithm regarding the image processing and imange filtering.\\
\section{Advantages of an ASIC based design}
An application-specific integrated circuit (ASIC), is an integrated circuit (IC) customized for a particular use, rather than intended for general-purpose use.\\
The advantages of an ASIC can be divided into four major areas: Unit
cost, performance, power consumption and flexibility. 
\subsection{Unit cost}
One of the biggest advantages of an ASIC based product with respect to a general purpose product is the lower unit cost once a certain
volume has been reached. Unfortunately the volume required to offset
the high NRE costs of an ASIC is very high which means that many
projects are never a candidate for ASICs.\cite{asci0}\\


\subsection{Performance}
Another advantages for using an ASIC is the higher performance. It has been done a comparison of various designs and it was found that an ASIC design was on average 3.2
times faster than an FPGA \cite{asic2}. \\

\subsection{Power consumption}
An ASIC architecture usually has lower power consumption
than a comparable FPGA. 
This is due to the reconfigurability of the FPGA: there is a lot of logic in an FPGA which is used only for configuration.
While the dynamic power consumption of the reconfiguration logic
is practically 0, all of the configuration logic contributes to the leakage
power.\cite{asci0}\\
%\section{ASIC vs General Purpose}
