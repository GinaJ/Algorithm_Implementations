\chapter{Synthesis Results}
After having properly tested the architecture of each hardware implementation, the following step is its synthesis to determine the maximum clock frequency, area and power consumption.\\
The chosen sizes are smaller because I used the \textit{Pentium 4 adder} and the \textit{booth multiplier} as adder and as multiplier. Due to their complexity, the synthesys requires much time.

\section{Integral Image}
\begin{itemize}
	\item  16 
	\item size = 8x8
\end{itemize}
\begin{center}
	\begin{tabular}{  p{4.2cm} | p{6.7cm} }
			
		\hline
	 & \quad \textbf{Integral Image}\\
	 & \quad Pipelined\\
		\hline
		Total cell area & \quad37055.540208 $ \mu m^2{} $\\

		Data arrival time & \quad 0.62 $ ns  $\\
		Internal Power & \quad	3.9107$ mW $\\
		Switching Power & \quad1.8435$ mW $\\
		Total Dynamic Power & \quad5.7542$ mW $\\
		Leakage Power & \quad0.3724 $ mW $ \\
		Total Power & \quad6.1266 $ mW $\\
		\hline
		
	\end{tabular}
\end{center}
\bigskip

\section{Discrete Cosine Transformation} \label{synDCT}
For this DCT's hadware implementation (\ref{DCTH}), I synthesized both the pipeline and not pipeline architecture. I also synthesized a small version of the LLM architecture (\ref{llmarch}).
The not pipelined version, consumes more power and take a little time more than the pipelined one. However it has a smaller area due to the absence of pipeline registers.
\subsection{DCT pipelined}
\begin{itemize}
\item  8 bits for integer part
\item 8 bits for fractional part
\item size = 4
\end{itemize}
\begin{center}
	\begin{tabular}{ p{4.2cm} | p{8cm} }
		
		\hline 
		 & \quad \textbf{Discrete Cosine Transformation}\\
		& \quad Pipelined\\
		
		\hline
		Total cell area & \quad 70428.471340$ \mu m^2{} $\\

		Data arrival time & \quad 1.83 $ ns $\\
		Internal Power & \quad 4.5759$ mW $\\
		Switching Power & \quad 3.4865$ mW $\\
		Total Dynamic Power & \quad 8.0625$ mW $\\
		Leakage Power&\quad  0.6491 $ mW $\\
		Total Power & \quad 8.7116$ mW $\\
		\hline
		
	\end{tabular}
\end{center}
\bigskip
\subsection{DCT not pipelined}
\begin{itemize}
	\item  8 bits for integer part
	\item 8 bits for fractional part
	\item size = 4
\end{itemize}
%\textcolor{red}{{\Large MODIFICARE, mettere osservazioni}}
\begin{center}
	\begin{tabular}{ p{5.2cm} | p{8cm} }
		
		\hline 
		& \quad \textbf{Discrete Cosine Transformation}\\
		& \quad Not Pipelined\\
		
		\hline
		Total cell area & \quad 69801.783318$ \mu m^2{} $\\
		
		Data arrival time & \quad 2.05 $ ns $\\
		Internal Power & \quad 5.1175$ mW $\\
		Switching Power & \quad 3.9349$ mW $\\
		Total Dynamic Power & \quad 9.0523$ mW $\\
		Leakage Power&\quad  0.6446 $ mW $\\
		Total Power  & \quad 9.6970$ mW $\\
		\hline
		
	\end{tabular}
\end{center}
\bigskip
\subsection{LLM DCT}
The LLM DCT architecture shows better performance in everything.\\
However we remember the complexity of the design for high order (subsection \ref{HLLM}).
\begin{itemize}
	\item  8 bits for integer part
	\item 8 bits for fractional part
	\item size = 4
\end{itemize}
\begin{center}
	\begin{tabular}{ p{4.2cm} | p{8cm} }
		
		\hline 
		& \quad \textbf{LLM DCT}\\
	
		
		\hline
		Total cell area & \quad 21937.344292$ \mu m^2{} $\\
		
		Data arrival time & \quad 1.70 $ ns $\\
		Internal Power & \quad 2.8416$ mW $\\
		Switching Power & \quad 2.3913$ mW $\\
		Total Dynamic Power & \quad  5.2329$ mW $\\
		Leakage Power&\quad  0.2063 $ mW $\\
		Total Power & \quad 5.4392 $ mW $\\
		\hline
		
	\end{tabular}
\end{center}
\bigskip
\section{Binomial Filter}
I synthesized both the implementation described in the subsection \ref{h1} and \ref{h2}.\\We can clearly see that the version 1 is better than the version 2 because it requires less computation, actually it leaves the border values unchanged.
\subsection{Binomial Filter v1}
\begin{itemize}
	\item  16 bits
	\item size = 4x4
\end{itemize}
\begin{center}
	\begin{tabular}{ p{5.2cm} | p{8cm} }
		
		\hline 
		& \quad \textbf{Binomial Filter v1}\\
		
		
		\hline
		Total cell area & \quad 3795.379296$ \mu m^2{} $\\
		
		Data arrival time & \quad 0.87 $ ns $\\
		Internal Power & \quad 1.5640 $ mW $\\
		Switching Power & \quad 1.0043$ mW $\\
		Total Dynamic Power & \quad 2.5684$ mW $\\
		Leakage Power&\quad  0.0405198 $ mW $\\
		Total Power  & \quad 2.6089$ mW $\\
		\hline
		
	\end{tabular}
\end{center}
\bigskip

\subsection{Binomial Filter v2}
\begin{itemize}
	\item  16 bits
	\item size = 4x4
\end{itemize}
\begin{center}
	\begin{tabular}{ p{5.2cm} | p{8cm} }
		
		\hline 
	& \quad \textbf{Binomial Filter v2}\\
		
		
		\hline
		Total cell area & \quad 8223.321808$ \mu m^2{} $\\
		
		Data arrival time & \quad 0.93 $ ns $\\
		Internal Power & \quad 3.3129 $ mW $\\
		Switching Power & \quad 2.0846$ mW $\\
		Total Dynamic Power & \quad 5.3975$ mW $\\
		Leakage Power&\quad  0.0870836 $ mW $\\
		Total Power  & \quad 5.4846$ mW $\\
		\hline
		
	\end{tabular}
\end{center}
\bigskip
\section{FIR}
\begin{itemize}
	\item  16 bits
	\item size = 8
\end{itemize}
\begin{center}
	\begin{tabular}{ p{5.2cm} | p{8cm} }
		
		\hline 
		& \quad \textbf{FIR Filter}\\
		
		
		\hline
		Total cell area & \quad 35128.637271$ \mu m^2{} $\\
		
		Data arrival time & \quad 2.55 $ ns $\\
		Internal Power & \quad2.3964 $ mW $\\
		Switching Power & \quad 1.8437$ mW $\\
		Total Dynamic Power & \quad 4.2401 $ mW $\\
		Leakage Power&\quad  0.3212 $ mW $\\
		Total Power  & \quad 4.5612$ mW $\\
		\hline
		
	\end{tabular}
\end{center}
\bigskip
\section{Transport Equation Problem}
\begin{itemize}
	\item  16 bits
	\item size = 3x3
\end{itemize}
\begin{center}
	\begin{tabular}{ p{5.2cm} | p{8cm} }
		
		\hline 
		\label{syn_tep}& \quad \textbf{Transport Equation Problem}\\
		
		
		\hline
		Total cell area & \quad 43544.045428$ \mu m^2{} $\\
		
		Data arrival time & \quad 1.88 $ ns $\\
		Internal Power & \quad8.9910 $ mW $\\
		Switching Power & \quad 7.1353$ mW $\\
		Total Dynamic Power & \quad 16.1263 $ mW $\\
		Leakage Power&\quad  0.4151 $ mW $\\
		Total Power  & \quad 16.5418$ mW $\\
		\hline
		
	\end{tabular}
\end{center}
\bigskip