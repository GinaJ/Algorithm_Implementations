\chapter{Thessa tool}
\section{Analysis on the Binomial filter algorithm}

The binomial filter implemented in this tools, is different from the one in this thesis.
The difference is that in this case it performs an arithmetic average as you can see above in C code
\lstset{ %Formatting for code in appendix
	language=C,
	basicstyle=\footnotesize,
	numbers=left,
	stepnumber=1,
	showstringspaces=false,
	tabsize=1,
	breaklines=true,
	breakatwhitespace=false,
}

\begin{lstlisting}[frame=single]  % Start your 

#include <stdio.h>
#define V 10000

int main(){
int w = 8;
int h = 8;
int in[h][w] = {{ 0,0,0,0,0,0,0,0},{ 0,1,6,6,1,1,1,0},{ 0,1,6,6,1,1,1,0},{ 0,1,6,6,1,1,1,0},{ 0,1,6,6,1,1,1,0},{ 0,1,6,6,1,1,6,0},{ 0,1,6,6,1,1,6,0},{ 0,0,0,0,0,0,0,0}};
int out[h][w];
int i;
int j;
int divisor = 8;

for (i = 1; i < h-1; i++){
for (j = 1; j < w-1; j=j+1){

out[i][j] = (in[i-1][j-1] + in[i-1][j] + in[i-1][j+1] + in[i][j-1] + in[i][j] + in[i][j+1] + in[i+1][j-1] + in[i+1][j] + in[i+1][j+1])/divisor;
}
}

return 0;
}
\end{lstlisting}
Having the C code, we have yo translate it in a lower language as the MiRcode and obtain the following code.

\begin{tabular}{l l l}
0   & &    ['\$s', 'i', '1'] \\
1     & &  ['+', '\_f1', 'i', '1']\\ 
2 & &      ['-', '\_f0', 'i', '1'] \\
3  & L3:& ['for', '$>$=', 'i', '7', 'goto', 'L0']\\ 
4 & &      ['\$s', 'j', '1'] \\
5    & &   ['+', '\_f3', 'j', '1'] \\
6       & &['-', '\_f2', 'j', '1'] \\
7   &L2:& ['for', '$>$=', 'j', '7', 'goto', 'L1'] \\
8   & &    ['+', '\_t0', 'in[i-1][j]', 'in[i-1][j-1]'] \\
9  & &     ['+', '\_t1', '\_t0', 'in[i-1][j+1]'] \\
10    & &  ['+', '\_t2', '\_t1', 'in[i][j-1]']\\
11& &   ['+', '\_t3', '\_t2', 'in[i][j]']\\
12& &   ['+', '\_t4', '\_t3', 'in[i][j+1]']\\
13& &   ['+', '\_t5', '\_t4', 'in[i+1][j-1]']\\
14& &   ['+', '\_t6', '\_t5', 'in[i+1][j]']\\
15 & &  ['+', '\_t7', '\_t6', 'in[i+1][j+1]']\\ 
16 & &  ['/', '\_t8', '\_t7', 'divisor'] \\
17 & &  ['\$s', 'out[i][j]', '\_t8'] \\
18 & &  ['+', '\_t9', 'j', '1'] \\
19 & &  ['\$s', 'j', '\_t9'] \\
20 & &  ['+', '\_f3', 'j', '1']\\
21 & &  ['-', '\_f2', 'j', '1'] \\
22 & &  ['goto', 'L2']
\end{tabular}\\

\begin{tabular}{l l l}
23  &L1:& ['+', '\_t10', 'i', '1']\\
24 & &  ['\$s', 'i', '\_t10'] \\
25 & &  ['+', '\_f1', 'i', '1']\\
26 & &  ['-', '\_f0', 'i', '1']\\
27 & &  ['goto', 'L3']\\
28  &L0:& ['\$s', 'reset', '0']\\

\end{tabular}\\

After that we have to understand the data and memory dependencies between each single instruction or some critical sections.
We obtain the following graph where each number represent a single instruction (as you can see in the previous MiRcode)

\begin{figure}[h!]
	\centering
	\includegraphics[width=\textwidth]{imm/tessa/PDGscc.pdf} 	\caption{
	} 
	\label{pdg}
\end{figure}
The SCC and the instructions linked in red lines have to be performed sequentially.