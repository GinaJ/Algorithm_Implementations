\chapter{References}
\label{chap1}

%%%%%%%%%%%%%%%%%%%%%%%%%%%%%%%%%%%%%%%%%%%%%%%%%%%%%%%%%%%
% you can organize a chapter using sections -> \section{Simulating an inverter}
% or subsections -> \subsection{simulating a particular type of inverter}

%%%%%%   First section

\begin{itemize}
	%INTEGRAL IMAGE
	\item  2013 by Yong Cao, Referencing UIUC ECE408/498AL Course Notes (Lecture 10 - Virginia Tech)
	\item Prefix sum - Parallel algorithm - Wikipedia
	\item Patrick Cozzi,University of Pennsylvania
	%CIS 565 - Spring 2011, Lecture 15 Summed Area Tables
	\item Qingqing Dang, Shengen Yan, Ren Wu, A fast integral image generation algorithm on GPUs, 2014 IEEE \\
	%ripetuto
		\item 2015, Mario Cofano, Mariagrazia Graziano, Maurizio Zamboni, Design of a Logic-in-Memory architecture for massive parallel algorithms
	%DCT

	\item PRACTICAL FAST 1-D DCT ALGORITHMS
	WITH 11 MULTIPLICATIONS, Christoph Loeffler, Adriaan Lieenberg, and George S. Moschytz
	\item LLM Integer Cosine Transform and Its Fast Algorithm 
	Chi-Keung Fong, Student Member, IEEE, and Wai-Kuen Cham, Senior Member, IEEE
	
	\item High Order Integer Transform Design for  Video Compression,Jie Dong, Member, IEEE, and Yan Ye, Member, IEEE\\  
	\item 2015, Filasieno Francesco, Garino Simone, Garlando Umberto, Progettazione di Sistemi a Basso Consumo : Array Sistolico Riconfigurabile
	%BINOMIAL FILTER
	\item 2015, Mario Cofano, Mariagrazia Graziano, Maurizio Zamboni, Design of a Logic-in-Memory architecture for massive parallel algorithms
	\item DIGITAL IMAGE FILTERING By Fred Weinhaus
	
%	FIR FILTER 
	\item Integrated system architecture, lecture notes by E. Raviola, 2017
	%TRANSPORT EQUATION PROBLEM
	%ripetuto
	%\item 2015, Mario Cofano, Mariagrazia Graziano, Maurizio Zamboni, Design of a Logic-in-Memory architecture for massive parallel algorithms
	
	%For the introduction
	\item A Guide to Numerical Methods for Transport Equations, Dmitri Kuzmin
	
		
	 
\end{itemize}
 % here is the reference to the figure below

% Below is shown how you can insert a figure. If you give a label to the figure, you can refer to the figure using \ref{figure_label} as shown above. 

	
	
